\documentclass[11pt]{article}

\usepackage{geometry}
\usepackage{booktabs}
\usepackage{graphicx}
\usepackage[final]{pdfpages}
\usepackage[stable]{footmisc}
\usepackage{threeparttable}
\usepackage{indentfirst}
\usepackage{minted}
\usepackage{listings}
\usepackage{xcolor}
\usepackage{subfigure}
\usepackage{amsmath}
\usepackage{amsfonts}
\usepackage{hyperref}
\usepackage{cleveref}
\crefformat{figure}{#2图~#1#3}
\crefrangeformat{figure}{图~(#3#1#4)\;\~{}\;(#5#2#6)}
\crefmultiformat{figure}{图~(#2#1#3)}{和~(#2#1#3)}{,(#2#1#3)}{和~(#2#1#3)}

% \setlength{\parindent}{0pt}
\geometry{top=25mm,bottom=25mm,left=25mm,right=25mm}
\lstset{
    basicstyle=\tt,
    basicstyle=\small,
    keywordstyle=\color{purple}\bfseries,
    rulesepcolor= \color{gray},
    breaklines=true,
    numbers=left,
    numberstyle= \small,
    commentstyle=\color{gray},
    frame=shadowbox
}

\hypersetup{
    hypertex=true,
    colorlinks=true,
    linkcolor=blue,
    filecolor=magenta,      
    urlcolor=cyan,
    anchorcolor=blue,
    citecolor=blue
}

\urlstyle{same}

\title{HW4}
\author{Du Rui}
\date{}

\begin{document}

% \maketitle
\quad \\[0.1in] 
1. Encode the following sentences as Description Logic formulas:

a) Grandparents are either grandfathers or grandmothers

b) You can't be both a grandfather and a grandmother

c) Every person has grandparents

d) Grandparents are people who have at least one child \\[0.1in]

\textbf{Answer 1:}

a) $Grandparents \equiv Grandfathers \sqcup Grandmothers$

b) $Person \sqsubseteq \neg (Grandfathers \sqcap Grandmothers) $

% \qquad $Person \sqsubseteq (\neg Grandfathers \sqcup Grandfathers) \sqcap ( Grandfathers \sqcup \neg Grandfathers) $

c) $Person \sqsubseteq \forall hasGrandparents.Person$

d) $Grandparents \sqsubseteq Person \sqcap \exists hasChild.Person$





\quad \\[0.1in] 
2. Translate the following sentence into ALC: \emph{A professor is a person who is an expert in at least one topic, and everything they say is smart.} \\[0.1in]

\textbf{Answer 2:} $
Professor \equiv Person \sqcap \exists expertAt.Topic \sqcap \forall hasSaid.Smart
$



\quad \\[0.1in] 
3. For each of the formulas 1-6 listed in the green box on slide 19, explain with a calculation why the given extension of that concept is correct. \\[0.1in]

\textbf{Answer 3:}

1) $Artwork^\mathcal{I} \cap (\Delta^\mathcal{I} - Sculpture^\mathcal{I})$

2) $\mathcal{I} = \emptyset $, iterate every entry in $painted^{\mathcal{I}}$, add the first element into $\mathcal{I}$  whose second element is in $Painting^{\mathcal{I}}$. 

3) we first calculate $\exists sculptured.Artwork$ following the same rules as mentioned in 2), then we calculate $\forall created.Sculpture$. We calculate the $created^I$ grouped by the first element, denoted as $created^J$; initialize $\mathcal{I} = \emptyset $, iterate every entry, such as $(michelangelo, (sixtChappel, david))$, in $created^J$, if the second entry minus $Sculpture^\mathcal{I}$ is empty, then we add the first element into $\mathcal{I}$. Finally, we take the intersection.

4) 
\begin{align}
    Artwork \sqcap \neg Sculpture  = \{ nightwatch, sixtChappel \} \\
    \exists created.(Artwork \sqcap \neg Sculpture)  = \{ rembrandt, michelangelo \}\\
    \forall created.Sculpture  = \{ \}\\
    \forall created.Sculpture \sqcap exists created.(Artwork \sqcap \neg Sculpture)  = \{ \}
\end{align}

5) 

\begin{align}
    \forall created.Painting  = \{ rembrandt\}\\
    \exists created.\top = \{ rembrandt, michelangelo, rodin\} 
\end{align}

6) follow the same rule as mentioned in 2), let us iterate every entry: nightwatch is a Painting, so we add "rembrandt" into the result set, and sixtChappel is also a Painting, so we add "michelangelo".




\end{document}

% \begin{table}[htbp]
%     \caption{常用的处理机调度策略}
%     \centering

%     \begin{threeparttable}
%         \begin{tabular}{cccccc}
%             \toprule
%             算法名称           & 主要适用范围       & 默认调度方式         \\
%             \midrule
%             先来先服务         & 作业调度\&进程调度 & 非抢占式             \\
%             短作业(进程)优先 & 作业调度\&进程调度 & 非抢占式             \\
%             高响应比优先       & 作业调度           & 非抢占式             \\
%             时间片轮转         & 进程调度           & 抢占式(不抢时间片) \\
%             多级反馈队列       & 进程调度           & 抢占式(抢占时间片) \\
%             \bottomrule
%         \end{tabular}

%         \zihao{-6}
%         \begin{tablenotes}
%             \item [*]   调度策略也就是调度算法
%         \end{tablenotes}

%     \end{threeparttable}
%     \qquad
% \end{table}

% \begin{figure}[htbp]
%     \centering
%     \includegraphics[height=550pt]{v1-class-compat.png}
%     \caption{UML类图(第二版)}
% \end{figure}

% \begin{minted}[mathescape,
%     linenos,
%     numbersep=5pt,
%     frame=lines,
%     gobble=4,
%     framesep=2mm]{Java}
%     public interface Observable {
%         void attachObserver(Observer o);

%         void detachObserver(Observer o);

%         void notifyObservers();
%     }
% \end{minted}

% \begin{lstlisting}[language={java},caption={收容队列(基于响应比的优先队列)}]
% private PriorityQueue<Task> queue = new PriorityQueue<>(new Comparator<Task>() {
%     @Override
%     public int compare(Task o1, Task o2) {
%         return (o2.getResponseRate(Clock.minutes) - o1.getResponseRate(Clock.minutes) > 0) ? (1) : (-1);
%     }
% });
% \end{lstlisting}